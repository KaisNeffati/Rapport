%
% File: chap01.tex
% Author: Victor F. Brena-Medina
% Description: Introduction chapter where the biology goes.
%
\let\textcircled=\pgftextcircled
\chapter{Design}
\label{chap:intro}

\initial{T}his chapter will be devoted to present the design stage of our system. 
Designing is a creative process that describes the manner in which our system works.
We will start by presenting our architecture design, then we move to the general conception and 
finally the detailed design.	

\section{ General design }
Our project is essentially based on two main parts as described before.
 In the next section, we present the architecture of the system.
\subsection{ Architectural design }
Architectural design refers to the high level structures of a software system, 
the discipline of creating such structures, and the documentation of these structures. 
These structures are needed to reason about the software system. 
Each structure comprises software elements, relations among them, and properties of both elements and relations.
It's about making fundamental structural choices which are costly to change once implemented. 
Architecture design choices include specific structural options from possibilities in the design of software. 
\subsubsection{ Big Data architectural design }
In the last chapter we fixed what we will be using in term of patterns and tools.Now,we are able to just
 start implementing our lambda architecture.
\begin{figure}[h!]
	\centering
	\includegraphics[height=0.3\textheight]{fig01/lambdaImplementation}
	\mycaption[Lambda architecture implementation.]{Lambda architecture implementation.}
	\label{fig:FilialesEtClients}
\end{figure}
\subsubsection{ Microservices architectural design }
After fixing all patterns and choosing which technologies we would like to implement , it's time to make the last step.
The next figure show our microservice application architecture in it's first versions:
\begin{figure}[h!]
	\centering
	\includegraphics[height=0.4\textheight]{fig01/MicroservicesArchitecture}
	\mycaption[filiales et clients de Sofrecom.]{Microservices architecture implemention.}
	\label{fig:FilialesEtClients}
\end{figure}

\subsubsection{ Deployment architectural design }
 We have designed our deployment architecture showing all containers communication and linking pipelines.
 The next two figures show the big data and the microservices deployment architecture.

 \begin{figure}[h!]
	\centering
	\includegraphics[height=0.3\textheight]{fig01/BigDatadeployment}
	\mycaption[Lambda architecture deployment.]{Lambda architecture deployment.}
	\label{fig:FilialesEtClients}
\end{figure}

\begin{figure}[h!]
	\centering
	\includegraphics[height=0.3\textheight]{fig01/MicroservicesDeployment}
	\mycaption[Microservices architecture deployment.]{Microservices architecture deployment.}
	\label{fig:FilialesEtClients}
\end{figure}
\newpage

\subsubsection{ Summary }

Finally, the next figure resume all patterns and architecture discribed before.

\begin{figure}[h!]
	\centering
	\includegraphics[height=0.8\textheight]{fig01/GlobalArchitecture}
	\mycaption[Architecture global de platfrom.]{Architecture global de platfrom.}
	\label{fig:FilialesEtClients}
\end{figure}

\section{ Detailed design }
This section will be devoted to the detailed design of our application, 
we start with the package diagram. 
Then we present the class diagram.
And finally we model the dynamic aspect of the system using sequence diagrams
\subsection{ Package Diagram}
A package diagram is a collection of static modeling elements (classes, packages, ...) that shows the structure of a model. It describes all the classes and their associations
\begin{figure}[h!]
	\centering
	\includegraphics[height=0.3\textheight]{fig01/packagediagram}
	\mycaption[Package diagram.]{Package diagram.}
	\label{fig:FilialesEtClients}
\end{figure}
Packages in the microservices architecture are totally separated . This is one of the benifit of this approach.
When creating a new fonctionaly just plug and pluck.
 \begin{itemize}
	\item Promotions package : this package handle promotions managment like creating , updating , searching or adding promotions.
	\item Account package : this package handle users managment like creating , updating , searching or adding promotions. This package
	expose a very important fonctionalty that allow updating user location. Users location are display in a heatMap. 
	\item Display analytics package : This package takes care of data analytics results and display it in a dashboard. Allowing
	users to see realtime analytics results.
\end{itemize}

\subsection{ Class Diagram }
After completing the preliminary design phase, 
we will now detail further this design by presenting the class diagrams of the different packages.
\begin{figure}[h!]
	\centering
	\includegraphics[height=0.4\textheight]{fig01/ClassDiagram}
	\mycaption[Class diagram.]{Class diagram.}
	\label{fig:FilialesEtClients}
\end{figure}

 \begin{itemize}
	\item Account class : This class is used to define the user informations. 
	It presents the Name, first name, email, login, passwords , essentially the longitude and latitude used for the
	heatMap chart ...
	\item Promotions class : This class is used to define the promotion informations. 
	It presents the name, partner name, status, type, description ... 
	\item Users by place class : this class is an abstract class that have two implemention which are Batch users by place and streaming
	users par place. It allows as to track users basing on locations and determine wich place is the most visited one.
	\textit{For exmaple : if i am having a supermaket , by putting beacons in the each store corner , i will be mesuring the users
	vist by place in real time .}
\end{itemize}

\subsection{Object Sequence Diagrams}

A sequence diagram is an interaction diagram that shows how objects operate with one another 
and in what order. 
It is a construct of a message sequence chart. 
A sequence diagram shows object interactions arranged in time sequence.
\newpage
\subsubsection{Display analytics sequence diagrams}
\label{subsec:subsec01}
\begin{figure}[h!]
	\centering
	\includegraphics[height=0.85\textheight]{fig01/DisplayAnalyticsSequenceDiagram}
	\mycaption[Display analytics sequence diagram.]{Display analytics sequence diagram.}
	\label{fig:FilialesEtClients}
\end{figure}
\newpage
\subsubsection{Update location sequence diagrams} :
\begin{figure}[h!]
	\centering
	\includegraphics[height=0.85\textheight]{fig01/UpdateLocation}
	\mycaption[Update location sequence diagrams.]{Update location sequence diagrams.}
	\label{fig:FilialesEtClients}
\end{figure}
\newpage
\section{ Conclusion }
Two main parts are waiting to be implemented, the first one is the lambda architecture and the data analytics strategy. the second one is the microservices approach and the application logic.

%=========================================================